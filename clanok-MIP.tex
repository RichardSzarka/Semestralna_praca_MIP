% Metódy inž7inierskej práce

\documentclass[10pt,twoside,slovak,a4paper]{article}

\usepackage[slovak]{babel}
%\usepackage[T1]{fontenc}
\usepackage[IL2]{fontenc} % lepšia sadzba písmena Ľ než v T1
\usepackage[utf8]{inputenc}
\usepackage{graphicx}
\usepackage{url} % príkaz \url na formátovanie URL
\usepackage{hyperref} % odkazy v texte budú aktívne (pri niektorých triedach dokumentov spôsobuje posun textu)

\usepackage{cite}
%\usepackage{times}

\pagestyle{headings}

\title{Možnosti a efektívnosť získania jazykových znalostí prostredníctvom
e-learningu\thanks{Semestrálny projekt v predmete Metódy inžinierskej práce, ak. rok 2020/21, vedenie: Ing. Fedor Lehocki }} % meno a priezvisko vyučujúceho na cvičeniach

\author{Richard Szarka\\[2pt]
	{\small Slovenská technická univerzita v Bratislave}\\
	{\small Fakulta informatiky a informačných technológií}\\
	{\small \texttt{xszarkar@stuba.sk}}
	}

\date{\small 7. október 2020} % upravte



\begin{document}

\maketitle

\begin{abstract}
Získavanie vedomostí cez internet je čoraz bežnejšie. Jednou z najžiadanejších oblastí v e-learningu je najmä získavanie jazykových znalostí. Na osvojenie si jazyka v kontaktnom vzdelávaní je potrebné precvičovať jazyk rôznymi metódami rovnako ako pri autonómnom vzdelávaní pomocou e-learningu. Cieľom tejto práce je definovať a porovnať metódy nadobúdania jazykových zručností prostredníctvom e-learningu. V práci si rozoberieme metódy, ako napríklad úlohy zadávané softwarom (Duolingo), aktívnu komunikáciu s osobou, ktorá daný jazyk už ovláda (Tandem, HelloTalk) alebo pasívnu komunikáciu so skupinou ľudí s rovnakým cieľom (jazykové blogy). Zameriame sa na to, aké výhody majú jednotlivé metódy, ktoré konkrétne jazykové zručnosti rozširujú, ale aj aká je efektívnosť nadobúdania jazykových znalostí cez e-learning.
\end{abstract}


\bibliography{zdroje_literatura}
\bibliographystyle{plain}
\end{document}
