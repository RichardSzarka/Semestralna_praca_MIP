% Metódy inž7inierskej práce

\documentclass[10pt,oneside,slovak,a4paper]{article}

\usepackage[slovak]{babel}
%\usepackage[T1]{fontenc}
\usepackage[IL2]{fontenc} % lepšia sadzba písmena Ľ než v T1
\usepackage[utf8]{inputenc}
\usepackage{graphicx}
\usepackage{url} % príkaz \url na formátovanie URL
\usepackage{hyperref} % odkazy v texte budú aktívne (pri niektorých triedach dokumentov spôsobuje posun textu)

\usepackage{cite}
%\usepackage{times}

\pagestyle{myheadings}

\title{Možnosti a efektívnosť získania jazykových znalostí prostredníctvom
e-learningu\thanks{Semestrálny projekt v predmete Metódy inžinierskej práce, ak. rok 2020/21, vedenie: Ing. Fedor Lehocki }} % meno a priezvisko vyučujúceho na cvičeniach

\author{Richard Szarka\\[2pt]
	{\small Slovenská technická univerzita v Bratislave}\\
	{\small Fakulta informatiky a informačných technológií}\\
	{\small \texttt{xszarkar@stuba.sk}}
	}

\date{\small 7. október 2020} % upravte



\begin{document}

\maketitle

\begin{abstract}
Získavanie vedomostí cez internet je čoraz bežnejšie. Jednou z najžiadanejších oblastí v e-learningu je najmä získavanie jazykových znalostí. Na osvojenie si jazyka v kontaktnom vzdelávaní je potrebné precvičovať jazyk rôznymi metódami rovnako ako pri autonómnom vzdelávaní pomocou e-learningu. Cieľom tejto práce je definovať a porovnať metódy nadobúdania jazykových zručností prostredníctvom e-learningu. V práci si rozoberieme metódy, ako úlohy zadávané softwarom (Duolingo), aktívnu komunikáciu s osobou, ktorá daný jazyk už ovláda (Tandem, HelloTalk) alebo pasívnu komunikáciu so skupinou ľudí s rovnakým cieľom (jazykové blogy). Zameriame sa na to, aké výhody i nevhody majú jednotlivé metódy a aká je efektívnosť nadobúdania jazykových znalostí cez e-learning.
\end{abstract}

\section{Úvod} %1 uvod
V súčastnosti nastal veľký rozmach v oblasti vzdelávania jazykových zručností prostredníctvom e-learningu. Mobilné aplikácie a jazykové blogy s stali veľmi obľúbenými najmä u mladých ľudí, a to predovšetkým vďaka svojej atraktivite jednoduchosti používania a možnosti motivácie.

Prekladaná práca pozostáva z 5 kapitol. V druhej kapitole si zadefinujeme pojem e-lerning a bližšie sa zameriame na jeho výhody a nevýhody. V tretej kapitole si rozoberieme najznámejšie a najoblúbenejšie aplikácie a formy vzdelávania sa v jazykoch. Konkrétne sa budeme venovať aplikácii Duolingo, HelloTalk, Tandem a jazykovým blogom. 4. kapitola slúži na posúdenie efektívnosti  e-learningu pri jazykovom vzdelávaní. Predmetom tejto kapitoly bude štúdia uskutočnená na Katedre Anglického vzdelávania Univerzity Borneo v Tarakane, ktorá skúmala vplyv aplikácie Duolingo na zlepšenie slovnej zásoby u 19-tich študentov po dobu 30 dní.

Hlavným cieľom predkladanej práce je porovnať najznámejšie možnosti vzdelávania sa v oblasti jazykov a posúdenie ich efektívnosti.

\section{Čo je e-learning?}%2
Učenie sa v dnešnej dobe je tak jednoduché, že naň stačí použiť hocijaké komunikačné zariadene, ktoré je schopné dávať informácie\cite{vyhody}.Pod pojmom e-learning rozumieme vedomé použitie kominikačných a sieťových technológií v učení a učení sa. Alternatívnou definiciou e-learningu môže byť aplikovanie elektronických systémov, ako internet a počítače, ktoré šetria čas aj financie \cite{efektivnost}. 

\subsection{Výhody e-learningu}%2.1
Ku každej novej technológii patria výhody, tak ako aj nevýhody. V prvej podkapitole sa zameriame na výhody e-learningu v učení a výučbe, ktoré zahrňujú aj výučbu jazykov. Medzi tie patrí najmä to, že:
\begin{itemize}
\item e-learning je rýchly, dynamický a znižuje výdavky (ako napríklad cestovanie), \cite{efektivnost}
\item lekcie sú pripravované rôznymi učiteľmi, \cite{efektivnost}
\item interaktívne cvičenia zvyšujú motiváciu študentov, \cite{vyhody}
\item aktivity e-learningu prinášajú rozličné skúsenosti rôznym ľuďom a takéto aktivity pomáhajú tiež ľahšiemu učeniu, \cite{vyhody}
\item je to druh kooperatívneho vzdelávania. \cite{efektivnost}
\end{itemize}

\subsection{Nevýhody e-learningu}%2.2
V druhej podkapitole sa budeme venovať nevýhodám jazykového vzdelávania prostredníctvom e-learningu. Medzi ne môžeme zaradiť napríklad:

\begin{itemize}
\item jazykové a kultúrne rozdiely, \cite{efektivnost}
\item technologické problémy u študentov, ako technofóbia či  nedostupnosť požadovaných technológií, \cite{nevyhody}
\item znížená sociálna a kultúrna interakcia, ako je napríklad reč tela, \cite{nevyhody}
\item technické limitácie.\cite{efektivnost}
\end{itemize}

\section{Najznámejšie a najobľúbenejšie e-learningové možnosti učenia sa jazyka}%3
Najväčší rozmach získavania jazykových vedomostí zažívajú v dnešnej dobe hlavne rôzne mobilné aplikácie. Mnohé z nich sú považované za edukatívne hry, ako napríklad Duolingo. Atraktivita týchto aplikácií spočíva v ľahkej dostupnosti a v tom, že ich používanie je dostupné zadarmo, s výnimkou občasných reklám a zakúpenia plnej (premium) verzie aplikácie. Avšak, v ústraní neostali ani jazykové blogy.

\subsection{Aplikácia Duolingo}%3.1
Duolingo je mobilná aplikácia na vzdelávanie jazykov. Jej používanie je jednoduché a je vhodná takmer pre všetky vekové kategórie\cite{duolingo}. Umožňuje získavanie jazykových znalostí pomocou rôznych kurzov. Užívateľ si vyberie jazyk, ktorým hovorí (ak je dostupný) a jazyk, ktorý by sa chcel naučiť, prípadne ten, v ktorom by sa chcel zdokonaliť. Aplikácia Duolingo momentálne ponúka 36 kurzov pre anglicky hovoriacich užívateľov. Ponuka sa bude líšiť v závislosti od materinského jazyka. Pre slovensky hovoriacich užívateľov napríklad nie je dostupný žiadny kurz a pre českých uživateľov disponuje Duolingo iba kurzom anglického jazyka. Ak si ako jazyk, ktorému rozumiete, vyberiete nemčinu, tak máte na výber 4 kurzy(\href{https://www.duolingo.com/courses}{www.duolingo.com/courses}).

Aplikácia postupne pridáva nové kurzy a aktívne zveľaduje obsah, čím zvyšuj svoj dosah i atraktivitu pre potenciánych užívateľov. Pre porovnanie, v roku 2012 poskytovalo Duolingo len 5 kurzov a v roku 2020 celkovo až 95 (\href{https://www.businessofapps.com/data/duolingo-statistics/}{www.businessofapps.com/data/duolingo-statistics}) Tento vývoj môžeme vidieť aj na množstve aktívnych užívateľov aplikácie \ref{duo-uzivatelia}.

\begin{figure}[h] %obrázok
\centering
\includegraphics[width=\textwidth]{duolingo.png}
\caption{ Aktívni užívatelia aplikácie Duolingo v miliónoch.
Zdroj: \href{https://www.businessofapps.com/data/duolingo-statistics/}{www.businessofapps.com/data/duolingo-statistics}}
\label{duo-uzivatelia}
\end{figure}

Na grafe vidíme, že v roku 2012 bola aplikácia sprístupnená verejnosti. Rok na to (2013) bol počet aktívnych užívateľov na úrovni 5 miliónov. V roku 2014 sa množstvo aktívnych užívateľov zdvojnásobilo na 10 miliónov. O 2 roky neskôr (2016) ich bolo už 20 miliónov. V ďalších rokoch (2018, 2019), že pribudlo v oboch prípadoch 5 miliónov aktívnych užívateľov.

Pomerne novou funkciou na rozvoj jazykových zručností je sekcia "Duolingo stories". (\href{https://www.duolingo.com/stories/}{www.duolingo.com/stories}). V krátkych príbehov, v ktorých musí užívateľ zadávať gramaticky správne odpovede alebo vybrať odpoveď, ktorá sa hodí do daného kontextu. Veľkou výhodou tejto formy vzdelávania je udržanie pozornosti pomocou nepredvídateľných a zaujímavých koncov príbehov. 

\subsection{Aplikácia HelloTalk} %3.2
HelloTalk je populárny na trhu aplikácií na vzdelávanie jazyka. jeho hlavnou myšlienkou je prepojenie užívateľa, ktorý sa chce naučiť cudzí jazyk s osobou, ktorej je daný jazyk materisnký. HelloTalk disponuje množstvom rôznych funkcií. Niektoré z nich sa odomknú až po zakúpení prémium verzie. Podľa webovej stránky HelloTalk má aplikácia až sedem miliónov užívateľov.\cite{hellotalk}\\
Aplikáciou môžete zdieľať hlasové správy, čety, mobilnú kameru, čmáranice, emotikony, GPS lokáciu či špecifické pomocné prvky na vzdelávanie jazyka (preklad, rozozanie hlasu, ...).\cite{hellotalk}\\\\
Výhody aplikácie HelloTalk:
\begin{itemize}
\item možnosť bezpatného hovoru,
\item automatické prekladanie konverzácie, čo zabezpečuje plynulosť koverzácie,
\item pomoc pri výslovnosti pri jazykoch, ktoré nepoužívaju latinskú abecedu.\cite{hellotalk}
\end{itemize}
Nevýhody aplikácie HelloTalk:
\begin{itemize}
\item nedostatočný systém na motivovanie užívateľa,
\item nie všetky funkcie su dostupné bez prémium verzie,
\item žiadna spätná väzba pre užívatelov o ich progese v učení,
\item informovanie o čase v krajine osoby, ktorej píšem (odpoveď na otázku, či je vhodné začať konverzáciu).\cite{hellotalk}
\end{itemize}
\subsection{Aplikácia Tandem} %3.3
Tandem je aplikácia založená na metóde výmeny jazykových znalostí, ktorá má korene v sedemdesiatych rokoch minulého storočia. Metológia vzdelávania aplikácie Tandem je založená na koncepte rozhovoru medzi dvomi Tandem partnermi, pričom ideálne je ak jeden z nich má daný jazyk ako materinský.

Zaujímavým prvkom tejto aplikácie je proces posúdenia nových uživateľov.\cite{tandem} Tandem je zameraný čisto na jazykové účely, a preto tento proces pomáha vyfiltrovať adeptov s inými zámermi. Proces posúdenia prebieha v podobe vyplnenia budúceho profilu užívateľa. Pri zadávaní profilovej fotografie vás aplikácia Tandem automaticky upozorní, ak fotografia nespĺňa požadované kritériá \ref{tandem-obmedzenia}. Čas čakania na odsúhlasenie sa líši. Niektorí užívatelia boli odsúhlasení za niekoľko dní, iní za zopár týždnov.

\begin{figure}[h] %obrázok
\centering
\includegraphics{tandem2.png}
\caption{Upozornenie po zadaní profilovej fotky, na ktorej nie je tvár. Zdroj: Vlastné spracovanie}
\label{tandem-obmedzenia}
\end{figure}

Tandem pozostáva z troch hlavných zložiek. Prvou je komunita, ktorá slúži na hľadanie vhodných Tandem partnerov. Keďže partneri sa nemôžu považovať za jazykových expertov, tak aplikácia umožňuje spojenie sa s platenými učiteľmi jazyka, čo je druhou zložkou. Poslednou sú čety, ktoré sú určené na prehľad a sumarizáciu konverzácií užívateľa.\\\\\\\\\\\\
Výhody aplikácie Tandem: 
\begin{itemize}
\item užívateľ si vie vybrať Tandem partnera podľa spoločných záujmov uvedených v profile,\cite{tandem}
\item Tandem partneri si vedia navzájom opraviť gramatiku prostredníctvom vstavanej funkcie,
\item Tandem umožňuje videochaty \cite{tandem} a audio-správy,
\item užívatelia môžu filtrovať to, kto vidí ich profil a prípadne prerušiť kontakt s nežiadúcou osobou. \cite{tandem}
\end{itemize}
Nevýhody aplikácie Tandem:
\begin{itemize}
\item užívateľ musí mať aspoň základné porozumenie jazyka, ináč sa s tandem partnerom nedorozumie (avšak môže využiť služby učiteľov), \cite{tandem}
\item veľa užívateľov využíva aplikáciu na zoznamovanie sa s novými ľuďmi, čo môže brániť vo vzdelávaní, \cite{tandem}
\item proces posudzovania profilu môže trvať dlho, veľa uživateľov sa na to sťažuje, \cite{tandem}
\end{itemize}

\subsection{Jazykové blogy} %3.4

Blogy a weblogy sú ľuďom známe už od vzniku svetovej počítačovej siete. Blogy sú online priestory na písanie, ktoré môžu byť upravené v okamihu a publikované verejne cez internetové prehliadače. Fungujú na princípe online denníkov, ktorých obsah autor môže aktualizovať obsah denne. V edukačnej sfére sú blogy veľmi  populárne, najmä u učitelov a žiakov pri učení sa jazyka. Tí sa o ne čoraz viac zaujímajú, pretože blogovanie sa zameriava na zdieľanie jednoduchého textu a vzájomného komentovania príspevkov. Používanie blogov sa stále považuje za relatívne novú metódu výučby a učenia sa cudzieho jazyka. \cite{blog-mif}\\
Vo svojej štúdii Miftachudin \cite{blog-mif} uvádza v sumarizácii 3 výhody a 3 nevýhody jazykových blogov.\\
\\
Výhody jazykových blogov:
\begin{itemize}
\item zlepšenie písacích schopností,
\item používanie blogov núti užívateľa čítať viac v danom jazyku,
\item blogy sú dobré médium na interakciu s ľudmi v danom jazyku. \cite{blog-mif}
\end{itemize}
Nevyhody jazykových blogov:
\begin{itemize}
\item chýba prvok na motivovanie užívateľov, čo vedie k nepravidelnému používaniu jazykových blogov,
\item niektorí užívatelia nemajú dosť sebaistoty na zdieľanie ich prác,
\item užívatelia môžu mať problém s porozumením inštrukcií na spravovanie blogu. \cite{blog-mif}
\end{itemize}



\section{Efektívnosť e-learningu pri jazykovom vzdelávaní}
V tejto sekcii si rozoberieme štúdiu zameranú na výskum efektivity jazykového vzdelávania prostredníctvom e-learningu. Konkrétne sa budeme venovať štúdii uskutočnenej na Katedre Anglického vzdelávania Univerzity Borneo v Tarakane.

\subsection{Vplyv Duolinga na slovnú zásobu}
V štúdii sa výsledky zlepšenia slovnej zásoby subjektov posudzovali pomocou testu na anglickú slovnú zásobu pred a po používaní aplikácie Duolingo. Subjektmi boli devätnásti študenti druhého semestra akademického roku 2018/2019 Katedry Anglického vzdelávania Unverzity Borneo v Tarakane. Ich úlohou bolo robiť zadania Duolinga, kým nedosiahli 20 bodov skúsenosti (body v Duolingu) po dobu 30 dní. Test pred aj po vyhradenej dobe pozostával z 25 otázok na slovnú zásobu. Výsledky testov môžeme vidieť na obrázku číslo \ref{duo-studium}. \cite{duolingo}

\begin{figure}[h] %obrázok
\centering
\includegraphics{duo_studium.png}
\caption{Výsledky pred a po používania aplikácii Duolingo\cite{duolingo}}
\label{duo-studium}
\end{figure}

V tabuľke \ref{duo-studium} potorujeme, že najemenší počet bodov na teste pred používaním Duolinga je 44 a najväčší 72. Najmenší počet bodov na teste po používaní aplikácie  Duolingo po dobu 30 dní je 64 a najvačší počet bodov je 92. Priemer testu pred používaní Duolinga je 57,47 bodov a priemer testu po používaní Duolinga je 79,15. \cite{duolingo} Na základe týchto výsledkov vieme posúdiť, že Duolingo zlepšilo slovnú zásobu študentov.

\section{Záver}

\bibliography{zdroje_literatura}
\bibliographystyle{plain}
\end{document}
