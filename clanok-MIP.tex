% Metódy inž7inierskej práce

\documentclass[10pt,oneside,slovak,a4paper]{article}

\usepackage[slovak]{babel}
%\usepackage[T1]{fontenc}
\usepackage[IL2]{fontenc} % lepšia sadzba písmena Ľ než v T1
\usepackage[utf8]{inputenc}
\usepackage{graphicx}
\usepackage{url} % príkaz \url na formátovanie URL
\usepackage{hyperref} % odkazy v texte budú aktívne (pri niektorých triedach dokumentov spôsobuje posun textu)

\usepackage{cite}
%\usepackage{times}

\pagestyle{myheadings}

\title{Možnosti a efektívnosť získania jazykových znalostí prostredníctvom
e-learningu\thanks{Semestrálny projekt v predmete Metódy inžinierskej práce, ak. rok 2020/21, vedenie: Ing. Fedor Lehocki }} % meno a priezvisko vyučujúceho na cvičeniach

\author{Richard Szarka\\[2pt]
	{\small Slovenská technická univerzita v Bratislave}\\
	{\small Fakulta informatiky a informačných technológií}\\
	{\small \texttt{xszarkar@stuba.sk}}
	}

\date{\small 15. december 2020} % upravte



\begin{document}

\maketitle

\begin{abstract}
Získavanie vedomostí cez internet je čoraz bežnejšie. Jednou z najžiadanejších oblastí v e-learningu je najmä získavanie jazykových znalostí. Na osvojenie si jazyka v kontaktnom vzdelávaní je potrebné precvičovať jazyk rôznymi metódami rovnako ako pri autonómnom vzdelávaní pomocou e-learningu. Cieľom tejto práce je definovať a porovnať metódy nadobúdania jazykových zručností prostredníctvom e-learningu. V práci si rozoberieme metódy, ako úlohy zadávané softwarom (Duolingo), aktívnu komunikáciu s osobou, ktorá daný jazyk už ovláda (Tandem, HelloTalk) alebo pasívnu komunikáciu so skupinou ľudí s rovnakým cieľom (jazykové blogy). Zameriame sa na to, aké výhody i nevhody majú jednotlivé metódy a aká je efektívnosť nadobúdania jazykových znalostí cez e-learning.
\end{abstract}

\section*{Úvod} %0 uvod
V súčasnosti nastal veľký rozmach v oblasti vzdelávania jazykových zručností prostredníctvom e-learningu. Mobilné aplikácie a jazykové blogy sa stali veľmi obľúbenými najmä u mladých ľudí, a to predovšetkým vďaka svojej atraktivite, jednoduchosti používania a možnosti motivácie.

Predkladaná práca pozostáva z 3 sekcií. V sekcii \ref{1} si zadefinujeme pojem e-lerning a bližšie sa zameriame na jeho výhody a nevýhody. V sekcii \ref{2} si rozoberieme najznámejšie a najobľúbenejšie aplikácie a formy vzdelávania sa v jazykoch. Konkrétne sa budeme venovať aplikácii Duolingo, HelloTalk, Tandem a jazykovým blogom. \ref{3}. sekcia slúži na posúdenie efektívnosti  e-learningu pri jazykovom vzdelávaní. Predmetom tejto sekcie bude štúdia uskutočnená na Katedre Anglického vzdelávania Univerzity Borneo v Tarakane, ktorá skúmala vplyv aplikácie Duolingo na zlepšenie slovnej zásoby u 19-tich študentov po dobu 30 dní.

Hlavným cieľom predkladanej práce je porovnať najznámejšie možnosti vzdelávania sa v oblasti jazykov a posúdenie ich efektívnosti.

\section{Čo je e-learning?}%1
\label{1}
Učenie sa v dnešnej dobe je tak jednoduché, že naň stačí použiť hocijaké komunikačné zariadene, ktoré je schopné dávať informácie\cite{vyhody}.Pod pojmom e-learning rozumieme vedomé použitie kominikačných a sieťových technológií v učení a učení sa. Alternatívnou definíciou e-learningu môže byť aplikovanie elektronických systémov, ako internet a počítače, ktoré šetria čas aj financie \cite{efektivnost}. 

\subsection{Výhody e-learningu}%1.1
Ku každej novej technológii patria výhody, tak ako aj nevýhody. V prvej podkapitole sa zameriame na výhody e-learningu v učení sa a výučbe, ktoré zahrňujú tiež výučbu jazykov. Medzi tie patrí najmä to, že e-learning je rýchly, dynamický a znižuje výdavky (ako napríklad cestovanie). \cite{efektivnost}
  Lekcie sú pripravované rôznymi učiteľmi, \cite{efektivnost}
 interaktívne cvičenia zvyšujú motiváciu študentov, \cite{vyhody}
 aktivity e-learningu prinášajú rozličné skúsenosti rôznym ľuďom a takéto aktivity pomáhajú tiež ľahšiemu učeniu. \cite{vyhody}
 Ide o druh kooperatívneho vzdelávania. \cite{efektivnost}

\subsection{Nevýhody e-learningu}%1.2
V druhej podkapitole sa budeme venovať nevýhodám jazykového vzdelávania prostredníctvom e-learningu. Medzi ne môžeme zaradiť napríklad
 jazykové a kultúrne rozdiely, \cite{efektivnost}
 technologické problémy u študentov, ako technofóbia či  nedostupnosť požadovaných technológií, \cite{nevyhody}
 znížená sociálna a kultúrna interakcia, ako je napríklad reč tela,\cite{nevyhody} a taktiež aj 
 technické limitácie.\cite{efektivnost}


\section{Najznámejšie a najobľúbenejšie e-learningové možnosti učenia sa jazyka}%2
\label{2}
Najväčší rozmach získavania jazykových vedomostí zažívajú v dnešnej dobe hlavne rôzne mobilné aplikácie. Mnohé z nich sú považované za edukatívne hry, ako napríklad Duolingo. Atraktivita týchto aplikácií spočíva v ľahkej dostupnosti a v tom, že ich používanie je dostupné zadarmo, s výnimkou občasných reklám a zakúpenia plnej (premium) verzie aplikácie. Avšak, v ústraní neostali ani jazykové blogy.

\begin{figure}[h] %obrázok
\centering
\includegraphics[width=0.75\textwidth,height=0.15\textheight]{metody.png}
\caption{Najznámejšie a najobľúbenejšie metódy učenia sa jazyka e-learningom}
\label{diagram}
\end{figure}


\subsection{Aplikácia Duolingo}%2.1
Duolingo je mobilná aplikácia na vzdelávanie jazykov. Jej používanie je jednoduché a je vhodná takmer pre všetky vekové kategórie\cite{duolingo}. Umožňuje získavanie jazykových znalostí pomocou rôznych kurzov. Užívateľ si vyberie jazyk, ktorým hovorí (ak je dostupný) a jazyk, ktorý by sa chcel naučiť, prípadne ten, v ktorom by sa chcel zdokonaliť. Aplikácia Duolingo momentálne ponúka 36 kurzov pre anglicky hovoriacich užívateľov. Ponuka sa bude líšiť v závislosti od materinského jazyka. Pre slovensky hovoriacich užívateľov napríklad nie je dostupný žiadny kurz a pre českých užívateľov disponuje Duolingo iba kurzom anglického jazyka. Ak si ako jazyk, ktorému rozumiete, vyberiete nemčinu, tak máte na výber 4 kurzy(\href{https://www.duolingo.com/courses}{www.duolingo.com/courses}).

Aplikácia postupne pridáva nové kurzy a aktívne zveľaduje obsah, čím zvyšuj svoj dosah i atraktivitu pre potenciálnych užívateľov. Pre porovnanie, v roku 2012 poskytovalo Duolingo len 5 kurzov a v roku 2020 celkovo až 95. (\href{https://www.businessofapps.com/data/duolingo-statistics/}{www.businessofapps.com/data/duolingo-statistics}) Tento vývoj môžeme vidieť aj na množstve aktívnych užívateľov aplikácie na obrázku \ref{duo-uzivatelia}.

\begin{figure}[h] %obrázok
\centering
\includegraphics[width=0.75\textwidth,height=0.3\textheight]{duolingo.png}
\caption{ Aktívni užívatelia aplikácie Duolingo v miliónoch.
Zdroj: \href{https://www.businessofapps.com/data/duolingo-statistics/}{www.businessofapps.com/data/duolingo-statistics}}
\label{duo-uzivatelia}
\end{figure}

Na grafe vidíme, že v roku 2012 bola aplikácia sprístupnená verejnosti. Rok na to (2013) bol počet aktívnych užívateľov na úrovni 5 miliónov. V roku 2014 sa množstvo aktívnych užívateľov zdvojnásobilo na 10 miliónov. O 2 roky neskôr (2016) ich bolo už 20 miliónov. V ďalších rokoch (2018, 2019) pribudlo v oboch prípadoch 5 miliónov aktívnych užívateľov.

Pomerne novou funkciou na rozvoj jazykových zručností je sekcia "Duolingo stories". (\href{https://www.duolingo.com/stories/}{www.duolingo.com/stories}). V krátkych príbehoch musí užívateľ zadávať gramaticky správne odpovede alebo vybrať odpoveď, ktorá sa hodí do daného kontextu. Veľkou výhodou tejto formy vzdelávania je udržanie pozornosti pomocou nepredvídateľných a zaujímavých koncov príbehov. 

\subsection{Aplikácia HelloTalk} %2.2
HelloTalk je populárny na trhu aplikácií na vzdelávanie jazyka. Jeho hlavnou myšlienkou je prepojenie užívateľa, ktorý sa chce naučiť cudzí jazyk s osobou, ktorej je daný jazyk materinský. HelloTalk disponuje množstvom rôznych funkcií. Niektoré z nich sa odomknú až po zakúpení prémium verzie. Podľa webovej stránky HelloTalk má aplikácia až sedem miliónov užívateľov.\cite{hellotalk}

Aplikáciou môžete zdieľať hlasové správy, čety, mobilnú kameru, čmáranice, emotikony, GPS lokáciu či špecifické pomocné prvky na vzdelávanie jazyka (preklad, rozozanie hlasu, ...).\cite{hellotalk}\\

 Medzi výhody aplikácie HelloTalk zaraďujeme 
 možnosť bezpatného hovoru,
 automatické prekladanie správ, čo zabezpečuje plynulosť koverzácie,
 pomoc pri výslovnosti pri jazykoch, ktoré nepoužívaju latinskú abecedu a
 informovanie o čase v krajine osoby, ktorej píšeme (odpoveď na otázku, či je vhodné začať konverzáciu). \cite{hellotalk}

Nevýhodami aplikácie HelloTalk sú
 nedostatočný systém na motivovanie užívateľa, skutočnosť, že
 nie všetky funkcie sú dostupné bez prémium verzie a
 žiadna spätná väzba pre užívateľov o ich progese v učení. \cite{hellotalk}
\subsection{Aplikácia Tandem} %2.3
Tandem je aplikácia založená na metóde výmeny jazykových znalostí, ktorá má korene v sedemdesiatych rokoch minulého storočia. Metológia vzdelávania aplikácie Tandem je založená na koncepte rozhovoru medzi dvomi Tandem partnermi, pričom ideálne je ak jeden z nich má daný jazyk ako materinský.

Zaujímavým prvkom tejto aplikácie je proces posúdenia nových užívateľov.\cite{tandem} Tandem je zameraný čisto na jazykové účely, a preto tento proces pomáha vyfiltrovať adeptov s inými zámermi. Proces posúdenia prebieha v podobe vyplnenia budúceho profilu užívateľa. Pri zadávaní profilovej fotografie vás aplikácia Tandem automaticky upozorní, ak fotografia nespĺňa požadované kritériá. Toto upozornenie môžeme vidieť na obrázku \ref{tandem-obmedzenia}. Čas čakania na odsúhlasenie sa líši. Niektorí užívatelia boli odsúhlasení za niekoľko dní, iní za zopár týždnov.

\begin{figure}[h] %obrázok
\centering
\includegraphics[width=0.6\textwidth,height=0.3\textheight]{tandem2.png}
\caption{Upozornenie po zadaní profilovej fotky, na ktorej nie je tvár.}
\label{tandem-obmedzenia}
\end{figure}

Tandem pozostáva z troch hlavných zložiek. Prvou je komunita, ktorá slúži na hľadanie vhodných Tandem partnerov. Keďže partneri sa nemôžu považovať za jazykových expertov, tak aplikácia umožňuje spojenie sa s platenými učiteľmi jazyka, čo je druhou zložkou. Poslednou sú čety, ktoré sú určené na prehľad a sumarizáciu konverzácií užívateľa.\cite{tandem}\\

 Medzi výhody aplikácie Tandem patrí možnosť výberu Tandem partnera na základe spoločných záujmov uvedených v profile.
Tandem partneri si môžu navzájom opraviť gramatiku prostredníctvom vstavanej funkcie.
Tandem taktiež umožňuje videochaty a audio-správy a
užívatelia môžu zvoliť, kto vidí ich profil, prípadne prerušiť kontakt s nežiadúcou osobou. \cite{tandem}

Nevýhodou aplikácie Tandem je skutočnosť, že
užívateľ musí mať aspoň základné porozumenie jazyka, ináč sa s Tandem partnerom nedorozumie (avšak môže využiť služby učiteľov).
Veľa užívateľov využíva aplikáciu na zoznamovanie sa s novými ľuďmi, čo môže brániť vo vzdelávaní.
Proces posudzovania profilu môže trvať dlho, na čo sa veľa užívateľov sťažuje. \cite{tandem}

\subsection{Jazykové blogy} %2.4

Blogy a weblogy sú ľuďom známe už od vzniku svetovej počítačovej siete. Blogy sú online priestory na písanie, ktoré môžu byť upravené v okamihu a publikované verejne cez internetové prehliadače. Fungujú na princípe online denníkov, ktorých obsah autor môže aktualizovať denne. V edukačnej sfére sú blogy veľmi  populárne, najmä u učiteľov a žiakov pri učení sa jazyka. Tí sa o ne čoraz viac zaujímajú, pretože blogovanie sa zameriava na zdieľanie jednoduchého textu a vzájomného komentovania príspevkov. Používanie blogov sa stále považuje za relatívne novú metódu výučby a učenia sa cudzieho jazyka. \cite{blog-mif}

Vo svojej štúdii Miftachudin \cite{blog-mif} uvádza 3 výhody a 3 nevýhody jazykových blogov.

Výhodou jazykových blogov je
 zlepšenie písacích schopností.
 Používanie blogov núti užívateľa čítať viac v danom jazyku a
blogy sú dobré médium na interakciu s ľudmi v danom jazyku. \cite{blog-mif}

Medzi nevýhody patrí
 chýbajúci prvok na motivovanie užívateľov, čo vedie k nepravidelnému používaniu jazykových blogov.
Niektorí užívatelia nemajú dostatok sebaistoty na zdieľanie ich prác a taktiež,
môžu mať užívatelia problém s porozumením inštrukcií na spravovanie blogu. \cite{blog-mif}



\section{Efektívnosť e-learningu pri jazykovom vzdelávaní}
\label{3}
V tejto sekcii si rozoberieme štúdiu zameranú na výskum efektivity jazykového vzdelávania prostredníctvom e-learningu. Konkrétne sa budeme venovať štúdii uskutočnenej na Katedre Anglického vzdelávania Univerzity Borneo v Tarakane.

\subsection{Vplyv Duolinga na slovnú zásobu}
V štúdii sa výsledky zlepšenia slovnej zásoby subjektov posudzovali pomocou testu na anglickú slovnú zásobu pred a po používaní aplikácie Duolingo. Subjektmi boli devätnásti študenti druhého semestra akademického roku 2018/2019 Katedry Anglického vzdelávania Unverzity Borneo v Tarakane. Ich úlohou bolo robiť zadania Duolinga, kým nedosiahli 20 bodov skúsenosti (body v Duolingu) po dobu 30 dní. Test pred aj po vyhradenej dobe pozostával z 25 otázok na slovnú zásobu. Výsledky testov môžeme vidieť v tabuľke číslo \ref{duo-studium}. \cite{duolingo}

\begin{table}[h] %obrázok
\centering
\includegraphics[width=\textwidth,height=0.35\textheight]{duo_studium.png}
\caption{Výsledky pred a po používaní aplikácie Duolingo\cite{duolingo}}
\label{duo-studium}
\end{table}

V tabuľke \ref{duo-studium} pozorujeme, že najmenší počet bodov na teste pred používaním Duolinga je 44 a najväčší 72. Najmenší počet bodov na teste po používaní aplikácie  Duolingo po dobu 30 dní je 64 a najvačší počet bodov je 92. Priemer testu pred používaní Duolinga je 57,47 bodov a priemer testu po používaní Duolinga je 79,15. \cite{duolingo} Na základe týchto výsledkov vieme posúdiť, že Duolingo zlepšilo slovnú zásobu študentov.

\section*{Záver}
Mobilné aplikácie a jazykové blogy sú veľmi efektívnym nástrojom na vzdelávanie sa v oblasti jazykov. Pre moderného človeka sú rozoberané možnosti e-learningu výhodné najmä z dôvodu svojej atraktivity, ľahkej dostupnosti a kreativity spracovania.

V \ref{1}. sekcii sme si zadefinovali pojem e-learning a poukázali sme na jeho výhody, ako aj nevýhody. V \ref{2}. sekcii sme si rozobrali najatraktívnejšie metódy vzdelávania sa v oblasti jazykov prostredníctvom e-learnigu. Okrem toho sme si vysvetlili princíp fungovania vybraných mobilných aplikácií a jazykového blogu. Zhrnuli sme si taktiež výhody a nevýhody jednotlivých metód. Sekcia \ref{3} slúžila na poukázanie efektivity a posúdenie výsledkov používania jednej z metód uvedených v sekcii \ref{2}. Na základe výsledkov skúmanej štúdie môžeme konštatovať, že aplikácia Duolingo má pozitívny vplyv na zlepšenie jazykových zručností užívateľov. 

Hlavný cieľ predkladanej práce, ktorým bolo porovnať najznámejšie možnosti vzdelávania sa v oblasti jazykov a posúdenie ich efektívnosti, sa nám podarilo splniť. Uviedli sme si jednotlivé princípy metód a ich výhody, ako i nevýhody, na základe čoho sme boli schopní posúdiť efektivitu najznámejších možností vzdelávania.
%reakcia na prednášky
\paragraph{Spoločenské súvislosti.}
Efektívna a sprístupnená výučba jazykov má určite veľký dopad na spoločnosť hlavne z dlhodobého hľadiska. Keďže väčšina metód je robená spôsobom hier, alebo je tam nejaký prvok na motivovanie užívateľa, tak užívateľ má väčšiu tendenciu sa daný jazyk naučiť. Ak budú ľudia vďaka e-learningu zdatnejší v cudzích jazykoch, jazykové bariéry určite poklesnú a aj averzia voči cudzincom u niektorých jedincov.
\paragraph{Historické súvislosti.}
Už v minulosti vznikal koncept výučby jazykov pomocou nejakých foriem technológií. Predávali sa rôzne CD s kurzmi. Taktiež koncept aplikácií ako Tandem a HelloTalk majú korene v minulosti (70. roky minulého storočia). Vtedy vznikol koncept, ktorý bol založený na rozhovore dvoch ľudí (jeden sa jazyk chce naučiť a druhému je materinský). Taktiež môžeme vidieť, že prístupnosť verejnosti sa rokmi zvýšila, keďže CD s kurzmi a pod. sa museli zakúpiť a dnešné aplikácie sú až na niektoré funkcie zadarmo.
\paragraph{Udržateľnosť a etika.}
Etika výučby jazyka pomocou e-learningu je veľmi dôležitá. V aplikácii Tandem sa často stáva, že užívatelia využívajú aplikáciu ako "zoznamku", a tým pádom aplikácia stráca svôj smer a poslanie, ktoré pôvodne mala. Tandem preto obsahuje proces posudzovania profilu, ktorý má zabrániť vstúpeniu užívateľom s inými zámermi do komunity. Taktiež v aplikácii Duolingo sa miestami stane, že algoritmus zadá na precvičenie vetu, ktorá by mohla byť pre niektorých jemne ofenzívna. 

\bibliography{zdroje_literatura}
\bibliographystyle{plain}
\end{document}
